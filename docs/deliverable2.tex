\documentclass[11pt]{article}
\usepackage{graphicx} % graphic import stuff
\usepackage[parfill]{parskip} % to start each parapraph will an empty line before
\usepackage{listings}
\usepackage{hyperref}
\usepackage{subfig}

%
% settings
%
\DeclareGraphicsExtensions{.pdf,.png,.jpg} % omits endings of graphics
\graphicspath{{./img/}} % the path to the graphics
%
% command for √ and x
%
\newcommand{\cmark}{\ding{51}}%
\newcommand{\xmark}{\ding{55}}%

%
% centers, scales and wraps a given graphic into a figure env
% % [1]: the scale factor 
% % [2]: the name of the image which will also be used with 'fig:' prefix as label
%%  [3] the caption of the figure
%
\newcommand{\cgraphic}[3]
{
	\begin{figure}[!ht]
		\begin{center}
		\includegraphics[width=#1]{#2}
		\end{center}
		\caption{#3}
		\label{fig:#2}
	\end{figure}
}%

%
% creates a subfloat of 3 figures
% % [1] [2]
% % [3] [4]
% % [5] [6]
% % [7} the caption of the figure
% % [8} the label of the figure
%
\newcommand{\sfigThree}[8]
{
	\begin{figure}[!ht]
	  \centering
	  \subfloat[#2]{\label{fig:#1}\includegraphics[width=0.3\textwidth]{#1}}
	  \subfloat[#4]{\label{fig:#3}\includegraphics[width=0.3\textwidth]{#3}}
	  \subfloat[#6]{\label{fig:#5}\includegraphics[width=0.3\textwidth]{#5}}
	  \caption{#7}
	  \label{fig:#8}
	\end{figure}
}%

%
% creates a subfloat of 3 figures
% % [1] [2]
% % [3] [4]
% % [5} the caption of the figure
% % [6} the label of the figure
%
\newcommand{\sfigTwo}[6]
{
	\begin{figure}[!ht]
		\centering
		\subfloat[#2]{\label{fig:#1}\includegraphics[width=0.5\textwidth]{#1}}
		\subfloat[#4]{\label{fig:#3}\includegraphics[width=0.5\textwidth]{#3}}
		\caption{#5}
		\label{fig:#6}
	\end{figure}
}%

%
% create a title page with indivdual title
%% [1] the individual title
%
\newcommand{\createTitlepage}[1] 
{
	\begin{titlepage}
		\centering
		\includegraphics[width=0.7\textwidth]{hska-logo}\par\vspace{1cm}
		{\scshape\LARGE University of Applied Sciences - Karlsruhe \par}
		\vspace{1cm}
		{\scshape\Large Distributed Information Systems Lab\par}
		\vspace{1.5cm}
		{\huge\bfseries #1\par}
		\vspace{2cm}
		{\Large\itshape Manuel Vogel (54541)  - voma1041@hs-karlsruhe.de \\ Felix Mohr (55687) - mofe1015@hs-karlsruhe.de \\ Jinghua Lin (54268) - liji1015@hs-karlsruhe.de\par}
		\vfill
		supervised by\par
		Prof~Dr.~Christian \textsc{Zirpins}
		\vfill
		{\large \today\par}
	\end{titlepage}
}

%opening
\author{Manuel Vogel, Felix Mohr, Jinghua Lin}

\begin{document}
	\createTitlepage{Deliverable 2 : Architecture draft of Microservices}
	
	\section{Microservice Architecture}
	\cgraphic{\textwidth}{ms_arch}{Microservice Architecture Proposal} After a code revision of the legacy Web-Shop, we identified three Micro- and two Core-Services as shown in Figure \ref{fig:ms_arch}. 
	The actions of each service are attached to the corresponding arrows.
	\subsection{Core Services}
		\begin{description}
			\item[User Service] Handles the registration, logging in/out of each user. Generates the API tokens. Provides a method to obtain the role of the user from its token.
			\item[Product Service] Handles the creation, deletion and retrieval of products.
			\item[Category Service] Handles the creation, deletion and retrieval of categories.
		\end{description}
	\subsection{Composite Services}
		\begin{description}
			\item[List Product Service] Offer the listing of products with their associated categories.
			\item[Modify Product Service] Handles the addition of products (they need to belong to a category) and the deletion of categories with all its associated products. Checks if the user has the rights/role to perform this action.
		\end{description}
	\subsection{API Services}
		\begin{description}
			\item[ZUUL Edge Server] Represents the facade routing the REST-Calls to the correct microservice. Does not expose all the internal API Endpoints, only a meaningfully subset.
		\end{description}
	
	 Additionally we decided \textit{not} to split the database in three separate databases (one for each mircoservice). So it won't scale in the first shot. 
	
	Due to the fact of a later implementation of the \texttt{OAuth} authentication mechanism, we add a basic header authentication in this early draft. So it will be easier later on to switch it.
	
	Due the to availability of different API documentation possibilities like Swagger\footnote{\url{https://swagger.io}}, Spring API Docs and RAML, we choose the first two frameworks to document our APIs. So we gained knowledge about the advantages and disadvantages of each framework and could share it later in the group.
	
	\section{API Documentation}
	The API documentations are attached to this document in a \texttt{zip}-file.
	
\end{document}
