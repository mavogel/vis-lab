\documentclass[11pt]{article}
\usepackage{graphicx} % graphic import stuff
\usepackage[parfill]{parskip} % to start each parapraph will an empty line before
\usepackage{listings}
\usepackage{hyperref}
\usepackage{subfig}

%
% settings
%
\DeclareGraphicsExtensions{.pdf,.png,.jpg} % omits endings of graphics
\graphicspath{{./img/}} % the path to the graphics
%
% command for √ and x
%
\newcommand{\cmark}{\ding{51}}%
\newcommand{\xmark}{\ding{55}}%

%
% centers, scales and wraps a given graphic into a figure env
% % [1]: the scale factor 
% % [2]: the name of the image which will also be used with 'fig:' prefix as label
%%  [3] the caption of the figure
%
\newcommand{\cgraphic}[3]
{
	\begin{figure}[!ht]
		\begin{center}
		\includegraphics[width=#1]{#2}
		\end{center}
		\caption{#3}
		\label{fig:#2}
	\end{figure}
}%

%
% creates a subfloat of 3 figures
% % [1] [2]
% % [3] [4]
% % [5] [6]
% % [7} the caption of the figure
% % [8} the label of the figure
%
\newcommand{\sfigThree}[8]
{
	\begin{figure}[!ht]
	  \centering
	  \subfloat[#2]{\label{fig:#1}\includegraphics[width=0.3\textwidth]{#1}}
	  \subfloat[#4]{\label{fig:#3}\includegraphics[width=0.3\textwidth]{#3}}
	  \subfloat[#6]{\label{fig:#5}\includegraphics[width=0.3\textwidth]{#5}}
	  \caption{#7}
	  \label{fig:#8}
	\end{figure}
}%

%
% creates a subfloat of 3 figures
% % [1] [2]
% % [3] [4]
% % [5} the caption of the figure
% % [6} the label of the figure
%
\newcommand{\sfigTwo}[6]
{
	\begin{figure}[!ht]
		\centering
		\subfloat[#2]{\label{fig:#1}\includegraphics[width=0.5\textwidth]{#1}}
		\subfloat[#4]{\label{fig:#3}\includegraphics[width=0.5\textwidth]{#3}}
		\caption{#5}
		\label{fig:#6}
	\end{figure}
}%

%
% create a title page with indivdual title
%% [1] the individual title
%
\newcommand{\createTitlepage}[1] 
{
	\begin{titlepage}
		\centering
		\includegraphics[width=0.7\textwidth]{hska-logo}\par\vspace{1cm}
		{\scshape\LARGE University of Applied Sciences - Karlsruhe \par}
		\vspace{1cm}
		{\scshape\Large Distributed Information Systems Lab\par}
		\vspace{1.5cm}
		{\huge\bfseries #1\par}
		\vspace{2cm}
		{\Large\itshape Manuel Vogel (54541)  - voma1041@hs-karlsruhe.de \\ Felix Mohr (55687) - mofe1015@hs-karlsruhe.de \\ Jinghua Lin (54268) - liji1015@hs-karlsruhe.de\par}
		\vfill
		supervised by\par
		Prof~Dr.~Christian \textsc{Zirpins}
		\vfill
		{\large \today\par}
	\end{titlepage}
}

%opening
\author{Manuel Vogel, Felix Mohr, Jinghua Lin}

\begin{document}
	\createTitlepage{Deliverable 4 : Re-Implementation of the Webshop and OAuth2 integration}
	\tableofcontents
	\newpage
	
	\section{Research}
	Before adding \texttt{OAuth2} to the project a research period war necessary to understand the concept and
	different modes of the authorization process. Beside the official documentation, open source project were
	considered to achieve a good foundation of knowledge to integrate the concept into the existing project.
	\subsection{Official documentation}	
	The official \texttt{Spring} documentation (which can be found \href{http://projects.spring.io/spring-security-oauth/docs/oauth2.html}{here}) handles base cases of the different 
	\texttt{grant\_types} but not the special case with a \texttt{Zuul} edge server in front of all the services. So a deeper understanding of the Spring Cloud documentation (which can be found \href{http://cloud.spring.io/spring-cloud-security/spring-cloud-security.html}{here}) was necessary to understand the \textit{magic} happening behind the scenes. 
	
	\subsection{Open source projects}
	For deepening the knowledge after the research, the following open source projects were considered
	\begin{enumerate}
		\item \href{https://github.com/zirpins/vsmlab/tree/master/oauth2-demo}{OAuth2-Demo} from Prof. Zirpins
		\item \href{https://github.com/kakawait/uaa-behind-zuul-sample}{uaa-behind-zuul} from \texttt{kakawait}
		\item \href{https://github.com/sqshq/PiggyMetrics/}{PiggyMetrics} from \texttt{sqshq}
		\item \href{https://github.com/rwinch/spring-security-0-to-4.0}{spring-security-0-to-4.0} from \texttt{rwinch}
	\end{enumerate}	
	
	\newpage
	\subsection{Adaption and proof of concept (PoC)}
	The project \texttt{uaa-behind-zuul} with adaptions from \texttt{PiggyMetrics} was considered as a base. But it needed to be enhanced to meet to required use-case of the lab. After the enhancement the control flow looked like the diagram of figure \ref{fig:flow-of-control}.
	\cgraphic{\textwidth}{flow-of-control}{The control flow of the login}
	
	\texttt{Zuul} redirects to itself and then to the \texttt{uaa-service} which is responsible for the token generation and validation. If the use is not authenticated, it is redirected to the \texttt{login} page from the \texttt{uaa-service}. This page on the other hand is displayed via the gateway. So the user only talks to the proxy/edge-server. After entering the credentials, the \texttt{uaa-service} verifies the user against the \texttt{user-service}, which is responsible for \texttt{CRUD} operations on users. This authorization uses the OAuth2 \texttt{grant\_type=authorization code}.
	
	Additionally for the connection to the legacy web-shop the OAuth2 \texttt{grant\_type= password} was necessary to be implemented in the PoC. It was achieved by combining a patch from the \texttt{resource-server} branch of the base project and logic used in the PiggyMetrics project. Furthermore the authorization based on roles like \texttt{ADMIN} oder \texttt{USER} was implemented to show \texttt{401} errors if the rights were not sufficient.
	
	\section{System Design}
	\cgraphic{\textwidth}{ms_arch_oauth2}{The redesigned system with OAuth2}
	After the redesign the structure of the system looks like figure \ref{fig:ms_arch_oauth2}. Everything is hidden begin the \texttt{Zuul}-proxy. The chosen OAuth2 grant\_type is \texttt{grant\_type= password}.
	
	The \texttt{Zuul}-proxy works as a \texttt{SSO} and \texttt{Resource-Server} at the same time. Calls without a \texttt{BEARER} token will be redirected to the \texttt{uaa-service} and call which have a token will be validate and then directed to the composite service.
	
	The \texttt{Category-Composite} which is internally \textit{load-balanced} handles all the calls to the core-services and performs the authorization of a call (if a user has sufficient rights). After this stage no further authorization is performed. 
	
    Because the old database scheme was kept, the old legacy shop still usable beside the new one.
	
	\section{Realization}
	\subsection{Legacy Webshop 2.0}
	In the legacy webshop the \texttt{Hibernate} dependencies and calls via the DAO were removed and replaced by call to the \texttt{Zuul} proxy. 
	\cgraphic{\textwidth}{webshop-oauth2}{OAuth2 Template generation in the webshop}
	The OAuth2 token is obtained with the \texttt{grant\_type= password} and then stored in 
	an \texttt{OAuth2RestTemplate} sending the required token on each call. This is show in figure \ref{fig:webshop-oauth2}.
	
	\subsection{Command line client}
	Within the PoC the command line client was used to obtain the token. This is also possible for the redesigned backend. 
	\cgraphic{\textwidth}{curls-oauth2}{\texttt{curl} statements}
	
	The commands are shown in figure \ref{fig:curls-oauth2}.
	
	\subsection{Sources}
	All the sources can be found here \url{https://github.com/mavogel/vis-lab}
	
\end{document}
